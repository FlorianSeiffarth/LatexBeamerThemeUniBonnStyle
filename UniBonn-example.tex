\documentclass[aspectratio=1610]{beamer}
\usepackage[utf8]{inputenc}
\usepackage[T1]{fontenc}
\title{Latex Beamer Bonn Style}
\date{\today}
\author{Author}

\usetheme{UniBonn}

\begin{document}
	\begin{frame}[plain]
	\titlepage
\end{frame}

\begin{framecenter}
\frametitle{Title}
\framesubtitle{The proof uses \textit{reductio ad absurdum}.} 
\end{framecenter}

\begin{framevertical}
	\frametitle{Title}
	\framesubtitle{The proof uses \textit{reductio ad absurdum}.} 
\end{framevertical}


\begin{framecontent}
	\frametitle{Table of contents}
	\framesubtitle{The content of the presentation} 
\end{framecontent}



\begin{frameblank}
Eine Folie ohne Titel
\end{frameblank}

\section{The first section}
\begin{frame} 
\frametitle{There Is No Largest Prime Number} 
\framesubtitle{The proof uses \textit{reductio ad absurdum}.} 
\begin{theorem}
	There is no largest prime number. \end{theorem} 
\begin{enumerate}
	\item<1-| alert@1> Suppose $p$ were the largest prime number. 
	\begin{enumerate}
		\item Subitem
	\end{enumerate}
	\item<2-> Let $q$ be the product of the first $p$ numbers. 
	\item<3-> Then $q+1$ is not divisible by any of them. 
	\item<1-> But $q + 1$ is greater than $1$, thus divisible by some prime
	number not in the first $p$ numbers.
\end{enumerate}
\end{frame}

\section{The second section}
\begin{frame}
\begin{block}{Block title}
	Inhalt...
\end{block}
\end{frame}

\section{The third section}
\begin{frame}{A longer title}
\begin{itemize}
\item one
\item two
\end{itemize}
\end{frame}

\begin{frameverticalcontent}
	\frametitle{Table of contents}
	\framesubtitle{The content of the presentation} 
\end{frameverticalcontent}


\end{document}